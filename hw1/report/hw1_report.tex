\documentclass[12pt, a4paper]{article}

\usepackage{xeCJK, float, graphicx, indentfirst}
\usepackage[left=1in, right=1in]{geometry}

\setCJKmainfont{標楷體}

\author{資工四 B04902131 黃郁凱}
\title{\vspace{-2cm} Homework 1 Report - PM2.5 Prediction}

\begin{document}

\maketitle

\begin{enumerate}
\item 請分別使用至少4種不同數值的learning rate進行training(其他參數需一致),對其作圖,並且討論其收斂過程差異。

\item 請分別使用每筆data9小時內所有feature的一次項(含bias項)以及每筆data9小時內PM2.5的一次項(含bias項)進行training,比較並討論這兩種模型的root mean-square error(根據kaggle上的public/private score)。\par
When I use all features to train, I get $10.06$ on public score. But, I get $7.84$ using only PM2.5 feature. The huge difference $2.22$ comes from some features, like \textbf{WIND\_DIREC} and \textbf{WD\_HR}. These features barely relate to \textbf{PM2.5} but having large invariance values such that prediction of linear regression deteriorates. Linear regression is sensitive to outlier or biased data, so those features influence the performance.\par
To make sure that the feature indeed influece the results, I dump the parameters of linear regression and find that the average parameter values of \textbf{WIND\_DIREC} is ten times more than \textbf{PM2.5}. The results tell that before training, make sure all the feature in training data is helpful.

\begin{figure}[H]
    \centering
    \includegraphics[scale=0.4]{PM2.5.png}
    \includegraphics[scale=0.4]{WIND_DIREC.png}
    \caption{The plots of features: PM2.5(left), WIND\_DIREC(right) of values versus time. I can see that the values of WIND\_DIREC are large, change rapidly, and has no direct relation compared with PM2.5.}
\end{figure}

\item 請分別使用至少四種不同數值的regulization parameter λ進行training(其他參數需一至),討論及討論其RMSE(traning, testing)(testing根據kaggle上的public/private score)以及參數weight的L2 norm。

\end{enumerate}

\end{document}
