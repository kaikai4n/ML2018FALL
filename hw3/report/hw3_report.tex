\documentclass[12pt, a4paper]{article}

\usepackage{xeCJK, float, graphicx, indentfirst, amsmath, amssymb, physics, multirow, caption, hyperref}
\usepackage[left=1in, right=1in]{geometry}

\setCJKmainfont{標楷體}

\author{資工四 B04902131 黃郁凱}
\title{\vspace{-2cm} Homework 3 Report - \\Image Sentiment Classification}

\begin{document}

\maketitle

\begin{enumerate}
\item 請說明你實作的 CNN model,其模型架構、訓練過程和準確率為何?\\
我主要使用三種類別的CNN模型架構,分別是仿Mobilenet、仿VGG和VGG改編。
使用原本的Mobilenet與VGG的成效不好,validation大約$60\%$,因此都有做不少的變形。
以下是我仿造的Mobilenet和VGG的模型架構:\\
\begin{table}[h]
\centering
\begin{tabular}{|c|l|l|}\hline
    model&Mobilenet & VGG\\ \hline
    \multirow{12}{*}{Convolution Block}&conv\_bn(  1,  16, 1)&conv\_bn(1,16,1)\\
    &conv\_bn( 16,  32, 1)&conv\_bn(16,32,1)\\
    &conv\_bn( 32,  32, 2)&max\_pool\\
    &conv\_dw( 32,  64, 1)&conv\_bn(32,64,1)\\
    &conv\_dw( 64,  64, 1)&conv\_bn(64,64,1)\\
    &conv\_dw( 64, 128, 2)&max\_pool\\
    &conv\_dw(128, 256, 2)&conv\_bn(64,128,1)\\
    &conv\_dw(256, 512, 2)&conv\_bn(128,128,1)\\ 
    &                     &max\_pool\\
    &                     &conv\_bn(128,256,1)\\
    &                     &conv\_bn(256,256,1)\\
    &                     &max\_pool\\ \hline
    \multirow{3}{*}{Linear Block}&linear(512*3*3, 512)&linear(256*3*3,256)\\
    &linear(512, 7)&linear(256,32)\\
    &&linear(32, 7)\\ \hline
\end{tabular}
\caption{表格中的conv\_bn是一般convolution的基本架構,詳細在figure \ref{fig:conv_bn},而conv\_dw是depthwise convolution,也就是Mobilenet提出的特殊結構,由兩個convolution組成,在figure \ref{fig:conv_dw}有詳細內容。仿造Mobilenet和VGG16建造的模型,最終可以得到約$66\%$的表現。}
\end{table}

\begin{figure}[h]
    \begin{align*}
        conv\_bn(in,out,stride) &= Sequetial(\\
        &Conv2d(in, out, 3, stride, 1),\\
        &BatchNorm2d(out),\\
        &relu())
    \end{align*}
    \caption{The basic batchnorm convolution block.}
    \label{fig:conv_bn}
\end{figure}

\begin{figure}[h]
    \begin{align*}
        conv\_bn(in,out,stride) &= Sequetial(\\
        &Conv2d(in, out, 3, stride, 1, groups=in),\\
        &BatchNorm2d(out),\\
        &relu(),\\
        &Conv2d(in, out, 1, 1, 0),\\
        &BatchNorm2d(out),\\
        &relu())
    \end{align*}
    \caption{The basic depth-wise convolution block.}
    \label{fig:conv_dw}
\end{figure}
Mobilenet和VGG16在Convolutional Block的部分和我調整過後有些微差異,他們在深度上更深,並在前半的filter數量較少,後半較多;我調整後將深度減少,並在前半段filter少的部分多做幾層,後半段大的filter層數減少。經過如此調整,正確率可以從原本的$60\%$進步到$66\%$。\\
由於Mobilenet和VGG都是用來訓練Imagenet專用的模型架構,Imagenet有一百多萬張相片,分類目標有1000個class;相比我們的task規模較小,兩萬多張相片分類7個class。人臉表情辨識的模型不需要如他們一樣龐大的模型,精簡的就可以達到很好的效果。因此經過前人模型的經驗,我自己研發了一套模型可以達到$68\%$準確率。
\begin{table}[h]
\centering
\begin{tabular}{|c|l|l|}\hline
    model&Mobilenet & VGG\\ \hline
    \multirow{12}{*}{Convolution Block}&conv\_bn(  1,  16, 1)&conv\_bn(1,16,1)\\
    &conv\_bn(16,32,1)\\
    &max\_pool\\
    &conv\_bn(32,64,1)\\
    &conv\_bn(64,64,1)\\
    &max\_pool\\
    &conv\_bn(64,128,1)\\
    &conv\_bn(128,128,1)\\ 
    &max\_pool\\
    &conv\_bn(128,256,1)\\
    &conv\_bn(256,256,1)\\
    &max\_pool\\ \hline
    \multirow{3}{*}{Linear Block}&linear(512*3*3, 512)&linear(256*3*3,256)\\
    &linear(256,32)\\
    &linear(32, 7)\\ \hline
\end{tabular}
\caption{表格中的conv\_bn是一般convolution的基本架構,詳細在figure \ref{fig:conv_bn},而conv\_dw是depthwise convolution,也就是Mobilenet提出的特殊結構,由兩個convolution組成,在figure \ref{fig:conv_dw}有詳細內容。仿造Mobilenet和VGG16建造的模型,最終可以得到約$66\%$的表現。}
\end{table}


\item 承上題,請用與上述 CNN 接近的參數量,實做簡單的 DNN model,其模型架構、訓練過程和準確率為何?試與上題結果做比較,並說明你觀察到了什麼?\\

\item 觀察答錯的圖片中,哪些 class 彼此間容易用混? 並說明你觀察到了什麼? [繪出 confusion matrix 分析]\\

\item CNN time/space complexity: For a. b. Given a CNN model as
\begin{figure}[ht]
    \centering
    \includegraphics{./q4.png}
\end{figure}\\
And for the c. given the parameter as:
kernel size = (k,k);
channel size = c;
input shape of each layer = (n,n);
padding = p;
strides = (s,s);
\begin{enumerate}
    \item How many parameters are there in each layer?\\

    \item How many multiplications/additions are needed for a forward pass(each layer).\\
    
    \item What is the time complexity of convolutional neural networks?\\

\end{enumerate}

\item PCA practice:Problem statement: Given 10 samples in 3D space.\\
(1,2,3), (4,8,5), (3,12,9), (1,8,5), (5,14,2), (7,4,1), (9,8,9), (3,8,1), (11,5,6), (10,11,7)\\
\begin{enumerate}
    \item What are the principal axes?\\

    \item Compute the principal components for each sample.\\

    \item Reconstruction error if reduced to 2D.(Calculate the L2-norm)\\

\end{enumerate}

\end{enumerate}

\end{document}
