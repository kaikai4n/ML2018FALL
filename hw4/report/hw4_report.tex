\documentclass[12pt, a4paper]{article}

\usepackage{xeCJK, float, graphicx, indentfirst, amsmath, amssymb, physics, multirow, caption, hyperref}
\usepackage[left=0.5in, right=0.5in]{geometry}

\setCJKmainfont{標楷體}

\author{資工四 B04902131 黃郁凱}
\title{\vspace{-2cm} Homework 4 Report - \\Malicious Comments Identification}

\begin{document}

\maketitle

\begin{enumerate}
\item 請說明你實作之 RNN 模型架構及使用的 word embedding 方法,回
報模型的正確率並繪出訓練曲線。請實作 BOW+DNN 模型,敘述你的模型架構,回報正確率並繪出訓練曲線。\par
\begin{enumerate}
    \item RNN\par
    先將一個句子的每個中文字轉換成one-hot的形式,經過\texttt{torch.nn.Embedding()},會將每個字轉成一個\texttt{Embedding Dimension}的vector,接著依序送入GRU中,等到GRU吃完最後一個字(<EOS>),取出其hidden state的vector,flatten過後送入一連串的fully connected layer,並output一個數值,經過\texttt{sigmoid}後就是該句子有惡意的機率。模型總參數為$1.62\cdot 10^{7}$模型的public testing accuracy為$75.2\%$,private為$75.1\%$。
    \begin{figure}[H]
        \centering
        \includegraphics[width=0.7\textwidth]{images/ML2018FALL_hw4_RNN.png}
        \includegraphics[width=0.7\textwidth]{images/learning_curve_rnn.jpg}
        \caption{上圖為WordEmbedding+RNN的模型架構,下圖為其learning curve。}
    \end{figure}
    \item DNN\par
    先將一個句子的每個中文字轉成one-hot的形式,將句子內所有的one-hot vector相加形成7000多維的integer vector,送入一連串的fully connected layer,並output一個數值,經過\texttt{sigmoid}後就是該句子有惡意的機率。模型總參數為$1.64\cdot 10^{7}$。模型的public testing accuracy為$74.8\%$,private為$74.5\%$。
    \begin{figure}[H]
        \centering
        \includegraphics[width=0.7\textwidth]{images/ML2018FALL_hw4_DNN.png}
        \includegraphics[width=0.7\textwidth]{images/learning_curve_dnn.jpg}
        \caption{上圖為BOW+DNN的模型架構,下圖為其learning curve。}
    \end{figure}
\end{enumerate}
\item 請敘述你如何 improve performance(preprocess, embedding, 架構等),
並解釋為何這些做法可以使模型進步。\par
\begin{itemize}
    \item 使用mean of word embedding當作FC的feature input\par
    因為word embedding照理來講可以學到字與字之間關係,因此將所有word的embedding加起來取平均的向量,可以某個程度上代表這個句子想傳達的意思,例如某個髒話的詞重複數遍,那個在這個取平均的vector就會與那個髒話詞較相近,某種程度上讓DNN有更多線索可以判斷是否帶有惡意。正確率從$74.81\%$進步到$74.93\%$。
    \item 使用mean of hidden state當作FC的feature input\par
    原本只使用了最後一個時間點的hidden state當作FC的input,然而取用所有平均的hidden state可以傳達更多的訊息,較能代表整個語句的意思;再者,RNN的一個弱點是經過長時間的input會忘掉較先前的段句子傳達的意思,因此會有資訊的損失,在此透過取平均來拿回更多丟失的資訊。正確率從$74.93\%$進步到$75.2\%$。
\end{itemize}
\item 請比較不做斷詞 (e.g., 以字為單位) 與有做斷詞,兩種方法實作出來的
效果差異,並解釋為何有此差別。\par
\begin{table}[H]
    \centering
    \begin{tabular}{|c|c|c|}\hline
        有無斷詞&public score&private score\\\hline
        無&75.2&75.1\\ \hline
        有&73.8&73.6\\ \hline
    \end{tabular}
    \caption{我表現最好的model: RNN,在有無做斷詞下的testing score比較結果。}
\end{table}
有無斷詞其實效果差不多,正確率不會有太大的影響,而沒有斷詞的效果略為好。造成這樣的原因,可能是斷詞的的方法可能使語意有誤,例如“我們在野生動物園玩”若切成“我們/在野/生動/物/園/玩“,可能使RNN判斷語意出現錯誤;再者切詞會使字詞頻率減少,進而影響model判斷力,例如“白爛”和“好白爛”會被切成不同的詞,而“好白爛”出現次數相對非常少,RNN model並無法在少量的training data中學會兩者意思是相近的,然而如果每個字切開,那麼RNN model先學到“好”有更加的意思,且“白爛”有惡意,所以下次讀到“好白爛”就學到這句是有惡意的句子,相較於jieba切詞,將所有字都切開讓RNN model有更多自由,可以學到中文真正的意思。

\item 請比較 RNN 與 BOW 兩種不同 model 對於” 在說別人白痴之前,先想
想自己” 與” 在說別人之前先想想自己,白痴” 這兩句話的分數(model output),並討論造成差異的原因。\par
\begin{table}[H]
    \centering
    \begin{tabular}{|c|c|c|}\hline
        model&first sentence&second sentence\\\hline
        RNN&0&1\\ \hline
        BOW&1&1\\ \hline
    \end{tabular}
\end{table}
RNN model可以辨別出兩個句子是否惡意,而BOW無法。因為BOW是統計字詞的出現頻率,語意資訊有時候會丟失,例如此兩句就是例子,同樣在句子中出現白痴兩個字,一者有惡意一者沒有,而BOW無法辨別出兩者。RNN model可以多萃取出字詞順序間的關係,能夠有更多資訊判別語意。

\item In this exercise, we will train a binary classifier with AdaBoost algorithm
on the data shown in the table. Please use decision stump as the base classifier. Perform
AdaBoost algorithm for $T = 3$ iterations. For each iteration ($t = 1, 2, 3$), write down
the weights $u^n_t$ used for training, the weighted error rate $\epsilon_t$, scaling coefficient $\alpha_t$, and the classification function $f_t(x)$. The initial weights $u^n_1$ are set to 1 ($n = 0, 1, \cdots, 9$). Please refer to the course slides for the definitions of the above notations. Finally, combine the three classifiers and write down the final classifier.
\begin{table}[H]
    \centering
    \begin{tabular}{c|cccccccccc}\hline
        x&0&1&2&3&4&5&6&7&8&9\\
        y&+&-&+&+&+&-&-&+&-&-\\ \hline
    \end{tabular}
\end{table}

let $sign(x) = 
\begin{cases}
    1&\text{, if } x \geq 0\\
    -1&\text{, if } x < 0\\
\end{cases}
$\par
\begin{enumerate}
    \item T = 1
    \begin{itemize}
        \item $u_1 = [1, 1, 1, 1, 1, 1, 1, 1, 1, 1]$
        \item $f_1(x) = -sign(x-5)$
        \item $\epsilon_1 = \frac{2}{10} = \frac{1}{5}$
        \item $\alpha_1 = ln\sqrt{\frac{1-\epsilon}{\epsilon}} \approx 0.693$
    \end{itemize}
    \item T = 2
    \begin{itemize}
        \item $u_2 = [0.5, 2, 0.5, 0.5, 0.5, 0.5, 0.5, 2, 0.5, 0.5]$
        \item $f_2(x) = sign(x-2)$
        \item $\epsilon_2 = \frac{0.5+0+0+0+0+0.5+0.5+0+0.5+0.5}{8} = 0.31$
        \item $\alpha_2 = ln\sqrt{\frac{1-\epsilon}{\epsilon}} \approx 0.394$
    \end{itemize}
    \item T = 3
    \begin{itemize}
        \item $u_3 = [0.742, 1.348, 0.337, 0.337, 0.337, 0.742, 0.742, 1.348, 0.742, 0.742]$
        \item $f_3(x) = -sign(x-1)$
        \item $\epsilon_3 = \frac{0+0+0.337+0.337+0.337+0+0+1.348+0+0}{7.417} = 0.318$
        \item $\alpha_3 = ln\sqrt{\frac{1-\epsilon}{\epsilon}} \approx 0.381$
    \end{itemize}
    Final classifier: $H(x) = sign(\sum\limits_{i=1}^3 \alpha_i f_i(x)) \approx sign(-0.693sign(x-5) + 0.394 sign(x-2) - 0.381 sign(x-1))$
    \begin{table}[H]
        \centering
        \begin{tabular}{c|cccccccccc}\hline
            x&0&1&2&3&4&5&6&7&8&9\\
            y&+&-&+&+&+&-&-&-&-&-\\ \hline
        \end{tabular}
    \end{table}\par
    error = $0.1$
\end{enumerate}

\item In this exercise, we will simulate the forward pass of a simple LSTM cell. Figure shows a single LSTM cell, where $z$ is the cell input, $z_i$, $z_f$, $z_o$ are the control inputs of the gates, $c$ is the cell memory, and $f$, $g$, $h$ are activation functions. Given an input $x$, the cell input and the control inputs can be calculated by
\begin{align*}
    z &= w\cdot x+b\\
    z_i &= w_i\cdot x+b_i\\
    z_f &= w_f\cdot x+b_f\\
    z_o &= w_o\cdot x+b_o\\
\end{align*}
Where $w$, $w_i$, $w_f$, $w_o$ are weights and $b$, $b_i$, $b_f$, $b_o$ are biases. The final output can be calculated by
$$y = f(z_o)h(c')$$
where the value stored in cell memory is updated by
$$c' = f(z_i)g(z) + cf(z_f)$$
Given an input sequence $x^t$ ($t = 1, 2, \cdots, 8$), please derive the output sequence $y^t$. The input sequence, the weights, and the activation functions are provided below. The initial value in cell memory is $0$. Please note that your calculation process is required to receive full credit.
\begin{figure}[H]
    \centering
    \includegraphics[width=\textwidth]{images/q6.png}
\end{figure}
\begin{tabular}{|l|p{1.8cm}|p{1cm}|p{1cm}|p{1cm}|p{1cm}|p{1cm}|p{1.8cm}|p{1cm}|p{1cm}|}\hline
    Time step & $z=g(z)$ & $z_i$ & $z_f$ & $z_o$ & $f(z_i)$ & $cf(z_f)$ & $c'=h(c')$ & $f(z_o)$ & $y$ \\ \hline
	1&3&90&10&-10&1.00&0.00&3.00&0.00&0.00\\ \hline
	2&-2&90&10&90&1.00&3.00&1.00&1.00&1.00\\ \hline
	3&4&190&-90&90&1.00&0.00&4.00&1.00&4.00\\ \hline
	4&0&90&10&90&1.00&4.00&4.00&1.00&4.00\\ \hline
	5&2&90&10&-10&1.00&4.00&6.00&0.00&0.00\\ \hline
	6&-4&-10&110&90&0.00&6.00&6.00&1.00&6.00\\ \hline
	7&1&190&-90&90&1.00&0.00&1.00&1.00&1.00\\ \hline
	8&2&90&10&90&1.00&1.00&3.00&1.00&3.00\\ \hline
\end{tabular}

\end{enumerate}
\end{document}
